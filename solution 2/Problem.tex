\documentclass[12pt,a4paper, flushleft]{article}
\usepackage[utf8]{inputenc}
\usepackage{amsmath,amssymb,amsthm}
\usepackage[T2A]{fontenc}
\usepackage[russian, english]{babel}
\usepackage{mathrsfs, dsfont} % специальные шрифты, по типу \mathscr или \dsfont
\usepackage{comment} %для многострочных комментариев
\usepackage{xcolor} %для гиперссылок в тексте и их цвета
\usepackage{hyperref}
\usepackage{graphicx}
\usepackage{wrapfig}
\usepackage{lipsum}
\usepackage{multicol}
\graphicspath{/home/cowberry/Documents/10M/SPTYM/pics/}
\usepackage[left=1cm,right=1cm,top=2cm,bottom=2cm]{geometry}	
\usepackage[most]{tcolorbox}
\definecolor{block-gray}{gray}{0.90} % уровень прозрачности (1 - максимум)
\newtcolorbox{myquote}{colback=block-gray,grow to right by=-25mm,grow to left by=-25mm, boxrule=1pt,boxsep=0pt,breakable}
\author{Anatoly Kochenyuk, $\langle team~name\rangle$}
\date{July 2019}
\newcommand{\horline}[1]{
		\begin{center}
			\begin{picture}(#1, 2)
				\line(1,0){#1}%
			\end{picture}
		\end{center}
	}
\title{
	\vspace{4cm}	
	\horline{440}	
	\begin{center}
		\begin{Huge}
			\textbf{\emph{Problem 2. A Graph Coloring Game}}
		\end{Huge}
	\end{center}	
	\vspace{-1.3cm}	
	\horline{460}
	%\includegraphics[scale=0.15]{abgrx.png}
}
\newtheorem{Def}{Definition}[section]
\newtheorem{Th}{Theorem}[section]
\newtheorem{Lm}{Lemma}[section] 
\newtheorem{Pb}{Problem}[section]
\newtheorem{St}{Statement}[section]
\newtheorem{Sl}{consequence}[section]
\newtheorem{Zm}{Note}[section]
\newtheorem{Con}{Condition}[section]
\usepackage{relsize}
\usepackage{array}   % for \newcolumntype macro
\newcolumntype{C}{>{$}c<{$}} 
\newcommand{\vel}{\mathlarger{\mathlarger{\upsilon}}}
\newcommand{\der}[1]{\overset{\cdot}{#1}}
\newcommand{\dder}[1]{\overset{\cdot \cdot}{#1}}
\newcommand{\Lim}[2]{\lim\limits_{#1\to #2}}
\newcommand{\Ch}[1]{\overset{#1}{=}}
\newcommand{\p}[1]{#1^{\prime}}
\newcommand{\pp}[1]{#1^{\prime\prime}}
\newcommand{\ol}[1]{\overline{#1}}
\newcommand{\oll}[1]{\overline{\overline{#1}}}
\newcommand{\ov}[2]{\overset{#1}{#2}}
\newcommand{\un}[1]{\underline{#1}}
\newcommand{\gr}[1]{\includegraphics[scale=0.2]{../pics/#1}}
\newcommand{\lr}[1]{\langle #1 \rangle}
\newcommand{\ceil}[1]{\left\lceil #1 \right\rceil}
\newcommand{\floor}[1]{\left\lfloor #1 \right\rfloor}
\usepackage{comment}
\let\oldrightarrow\rightarrow
\renewcommand{\rightarrow}{%
  \mathrel{\raisebox{14pt}{$\oldrightarrow$}}%
}

\begin{document}
\maketitle
\vspace{4cm}
	
	\begin{myquote}
	\begin{center}
		\textbf{Annotation}\\
		\textit{
			$\lr{annotation~text}$
		}
	\end{center}
	\end{myquote}	
	
	\pagebreak

	\tableofcontents	
	
	\pagebreak
	
\section*{Introduction}
The main point of this problem is game theory.

In this case, game is played on an undirected graph with all vertices colored in white. Two payers, Bob and Riley, make alternating moves starting with Bob. On his first move, each player chooses
any white vertex and colors it in his favourite color. On each consequent move, each player picks
a white vertex connected with an edge to a vertex of his color (if there is no such vertex, the
player skips his move). Then he colors it in his color. The game finishes when no white vertices
remain.

\begin{Def}
	$G$ is an undirected Graph

	Let $B(G)$ be the maximal number of blue vertices, which Bob can guarantee at the end of
the game.

	Let $R(G)$ be the maximal number of red vertices, which Riley can guarantee at the
end of the game.
\end{Def}

\begin{Def}
	$M_B(n) = \min\limits_{G\in S(n)}B(G)\quad S(n)$ -- set of all simple undirected connected graphs with n vertices.
\end{Def}

\begin{Def}
	$|G|$, where $G$ is undirected graph, equals the number of vertices in $G$
\end{Def}

\section{Describe the optimal strategy for Bob and evaluate $B(G)$ where:}

\subsection{Tree} $G$ is a tree. Let's define weight for each vertex as following:
	
	\begin{Def}
		Let $S_v$ be a set of trees that was produced by removing vertex $v\in G$ from the tree
		
		Weight of vertex $v$ is $w_v = \max\limits_{s\in S}|s|$
	\end{Def}
	
	Now, let's take a look at Bob's first move. By choosing one vertex to color, he divides the tree into several subtrees. For Riley it's only possible to color one of these subtrees, because once he chooses one of these subtrees he can't 'move' to another one, because it's a tree and the only way goes through the vertex chosen by Bob. 
	
	So Riley chooses the subtree with the maximal number of vertices. And all the other subtrees goes to Bob.
	
	Thus, for Bob, it's beneficial to choose the vertex with the least weight for first move. 
	
	Because Riley 'takes' the subtree with the maximal number of vertices, $$B(G) = |G| - \min\limits_{v\in G} w_v$$
	
\subsection{Grid} $G$ is a grid $M\times K$. Let $M\leqslant K$. Then let's take a look at two different case:
	\begin{itemize}
		\item K is even. Then we can split the grid into two equal parts the following way:\\
		\begin{tabular}{CCCCCC}
			\gr{grid1} & \rightarrow & \gr{grid2} & \rightarrow & \gr{grid3} & \rightarrow\\
			\gr{grid4}& \rightarrow & \gr{grid5}	
		\end{tabular}		
		
		\begin{Def}
			Let \textbf{center line} be a shortest line that is placed in-between two edges of the grid, e.g. the following grid has two of them (colored vertical lines). 
			
			\gr{grid4}
		\end{Def}		
		
		Here appears invisible line splitting the grid in half. Bob's first move must be some vertex on a center line, because either way Riley can border the grid like shown on the next pictures:
		
		\begin{tabular}{CCCCCC}
			\gr{grid1} & \rightarrow & \gr{grid2_1} & \rightarrow & \gr{grid2_2.png} & \rightarrow\\
			\gr{grid2_3}
		\end{tabular}
		
		horizontal lines in the previous example could be center lines, but first move on them does not guarantee the optimal number of blue vertices.
		
		Therefore, in this case $B(G) = min(M, K) \cdot \dfrac{max(M, K)}{2}$, where $max(M, K)$ is even
		\item K is odd.
		
		\begin{itemize}
			\item M is odd. Then there's a center vertex. Bob's first move must be this vertex, i'll explain later why. We can draw four lines that border the grid into parts that include the vertex and the parts that 
don't.

			\gr{b} \hfill \gr{b2}

			By acting symmetrically Riley can border one of these areas. We know that $M~and~K$ are odd, then $\exists m, k: M = 2m+1\quad K = 2k+1$. 
			
			Areas of the areas are $(2m+1)\cdot (k+1) ~~\&~~ (2k+1)\cdot (m+1)$, i.e. $2mk+k + 2m + 1~~\&~~ 2mk + m + 2k + 1$. 
			
			$M\leqslant K\Rightarrow m\leqslant k\Rightarrow m + 2mk + m + k + 1\leqslant  k + 2mk + m + k\Rightarrow $
			
			$\Rightarrow (2m+1)(k+1)\leqslant (2k+1)(m+1)$, so Riley will choose to take one of  $2mk + k$ areas. 
			
			if first Bob's move was not the central vertex, then several areas would expand, increasing the gain of Riley.
			
			Therefore $B(G) = min(M, K)\cdot \ceil{\dfrac{max(M, K)}{2}}$
			
			\item M is even. Then there's a single center line. Bob's first move must be on this center line.
			
			\gr{b3}
			
			And after his move Riley can make his moves symmetrically, taking the other half.
			
			If bob's first move was somewhere not on the center line, then Riley could border the grid so that he gets most of area.
			
			Therefore $B(G) = \ceil{\dfrac{min(M, K)}{2}}\cdot max(M, K),$ where $min(M, K)$ is even.
		\end{itemize}
	\end{itemize}
	
	
	
	Let's sum it up using the fact that $M,\leqslant K$ and make some inequalities:
	\begin{itemize}
		\item $K$ is even $\quad B(G) = M \cdot \ceil{\dfrac{K}{2}} = M\cdot \dfrac{K}{2}  = \dfrac{M}{2}\cdot K\leqslant \ceil{\dfrac{M}{2}}\cdot K$
		\item $K$ is odd:
		\begin{itemize}
			\item $M$ is even $\quad B(G) = \ceil{\dfrac{M}{2}}\cdot K = \dfrac{M}{2}\cdot K = M\cdot \dfrac{K}{2}\leqslant M\cdot \ceil{\dfrac{K}{2}}$	
			\item $M$ is odd $\quad B(G) = M\cdot \ceil{\dfrac{K}{2}}\leqslant \ceil{\dfrac{M}{2}}\cdot K$
			\begin{proof}
				$M, K$ are odd $\Rightarrow \begin{cases} M = 2m+1 & \ceil{\dfrac{M}{2}} = m+1\\K = 2k+1& \ceil{\dfrac{K}{2}} = k+1\\ \end{cases}\quad M\leqslant K\Rightarrow m\leqslant k$
				
				$m\leqslant k\Rightarrow 2mk + k + 2m + 1\leqslant 2mk + 2k + m + 1\Rightarrow (2m+1)(k+1)\leqslant (m+1)(2k+1) \Rightarrow M\ceil{\dfrac{K}{2}}\leqslant \ceil{\dfrac{M}{2}}\cdot K$
			\end{proof}
		\end{itemize}		 
	\end{itemize}
	
	We can conclude, that $B(G) = min\left( \ceil{\dfrac{M}{2}}\cdot K~M\cdot \ceil{\dfrac{K}{2}}\right)$
	
\subsection{Torus Grid}

Since all vertices on a blank torus grid are completely identical Bob can start his game with any vertex.

Let's consider the same cases. Suppose that $G$ is a torus grid $M\times K$, where $M\leqslant K$:
\begin{itemize}
	\item $K$ is even
\end{itemize}  
\end{document}











































