\documentclass[12pt,a4paper, flushleft]{article}
\usepackage[utf8]{inputenc}
\usepackage{amsmath,amssymb,amsthm}
\usepackage[T2A]{fontenc}
\usepackage[russian, english]{babel}
\usepackage{mathrsfs, dsfont} % специальные шрифты, по типу \mathscr или \dsfont
\usepackage{comment} %для многострочных комментариев
\usepackage{xcolor} %для гиперссылок в тексте и их цвета
\usepackage{hyperref}
\usepackage{graphicx}
\usepackage{wrapfig}
\usepackage{lipsum}
\usepackage{multicol}
\graphicspath{/home/cowberry/Documents/10M/SPTYM/pics/}
\usepackage[left=1cm,right=1cm,top=2cm,bottom=2cm]{geometry}	
\usepackage[most]{tcolorbox}
\definecolor{block-gray}{gray}{0.90} % уровень прозрачности (1 - максимум)
\newtcolorbox{myquote}{colback=block-gray,grow to right by=-25mm,grow to left by=-25mm, boxrule=1pt,boxsep=0pt,breakable}
\author{Anatoly Kochenyuk, $\langle team~name\rangle$}
\date{July 2019}
\newcommand{\horline}[1]{
		\begin{center}
			\begin{picture}(#1, 2)
				\line(1,0){#1}%
			\end{picture}
		\end{center}
	}
\title{
	\vspace{4cm}	
	\horline{440}	
	\begin{center}
		\begin{Huge}
			\textbf{\emph{Problem 2. A Graph Coloring Game}}
		\end{Huge}
	\end{center}	
	\vspace{-1.3cm}	
	\horline{460}
	%\includegraphics[scale=0.15]{abgrx.png}
}
\newtheorem{Def}{Definition}[section]
\newtheorem{Th}{Theorem}[section]
\newtheorem{Lm}{Lemma}[section] 
\newtheorem{Pb}{Problem}[section]
\newtheorem{St}{Statement}[section]
\newtheorem{Sl}{consequence}[section]
\newtheorem{Zm}{Note}[section]
\newtheorem{Con}{Condition}[section]
\usepackage{relsize}
\usepackage{array}   % for \newcolumntype macro
\newcolumntype{C}{>{$}c<{$}} 
\newcommand{\vel}{\mathlarger{\mathlarger{\upsilon}}}
\newcommand{\der}[1]{\overset{\cdot}{#1}}
\newcommand{\dder}[1]{\overset{\cdot \cdot}{#1}}
\newcommand{\Lim}[2]{\lim\limits_{#1\to #2}}
\newcommand{\Ch}[1]{\overset{#1}{=}}
\newcommand{\p}[1]{#1^{\prime}}
\newcommand{\pp}[1]{#1^{\prime\prime}}
\newcommand{\ol}[1]{\overline{#1}}
\newcommand{\oll}[1]{\overline{\overline{#1}}}
\newcommand{\ov}[2]{\overset{#1}{#2}}
\newcommand{\un}[1]{\underline{#1}}
\newcommand{\gr}[1]{\includegraphics[scale=0.2]{../pics/#1}}
\newcommand{\Gr}[1]{\includegraphics[scale=0.3]{../pics/#1}}
\newcommand{\lr}[1]{\langle #1 \rangle}
\newcommand{\ceil}[1]{\left\lceil #1 \right\rceil}
\newcommand{\floor}[1]{\left\lfloor #1 \right\rfloor}
\usepackage{comment}
\let\oldrightarrow\rightarrow
\renewcommand{\rightarrow}{%
  \mathrel{\raisebox{14pt}{$\oldrightarrow$}}%
}

\begin{document}
\maketitle
\vspace{4cm}
	
	\begin{myquote}
	\begin{center}
		\textbf{Annotation}\\
		\textit{
			$\lr{annotation~text}$
		}
	\end{center}
	\end{myquote}	
	
	\pagebreak

	\tableofcontents	
	
	\pagebreak
	
\section*{Introduction}
The main point of this problem is the game theory.

In this case, the game is played on an undirected graph with all vertices colored in white. Two players, Bob and Riley, make alternating moves starting with Bob. On his first move, each player chooses
any white vertex and colors it in his favourite color. On each consequent move, each player picks
a white vertex connected by its edge to a vertex of his color (if there is no such vertex, the
player skips his move). Then he colors it in his color. The game finishes when no white vertices
remain.

\begin{Def}
	$G$ is an undirected Graph

	Let $B(G)$ be the maximum number of blue vertices that Bob can guarantee at the end of
the game.

	Let $R(G)$ be the maximum number of red vertices that Riley can guarantee at the
end of the game.
\end{Def}

\begin{Def}
	$M_B(n) = \min\limits_{G\in S(n)}B(G)\quad S(n)$ is the set of all simple undirected connected graphs with n vertices.
\end{Def}

\begin{Def}
	$|G|$, where $G$ is an undirected graph, equals the number of vertices in $G$
\end{Def}

\section{Describe the optimal strategy for Bob and evaluate $B(G)$ where:}

\subsection{Tree} $G$ is a tree. Let's define the weight for each vertex as follows:
	
	\begin{Def}
		Let $S_v$ be a set of trees that was produced by removing vertex $v\in G$ from the tree
		
		Weight of vertex $v$ is $w_v = \max\limits_{s\in S}|s|$
	\end{Def}
	
	Now, let's take a look at Bob's first move. By choosing one vertex to color, he divides the tree into several subtrees. For Riley it's only possible to color one of these subtrees, because once he chooses one of these subtrees he can't 'move' to another one, since it's a tree and the only way goes through the vertex chosen by Bob. 
	
	So Riley chooses the subtree with the maximal number of vertices. And all the other subtrees go to Bob.
	
	Thus, for Bob, it's beneficial to choose the vertex with the smallest weight for the first move. 
	
	Because Riley 'takes' the subtree with the maximum number of vertices, $$B(G) = |G| - \min\limits_{v\in G} w_v$$
	
\subsection{Grid} $G$ is a grid $M\times K$. Let $M\leqslant K$. Then let's take a look at two different cases:
	\begin{itemize}
		\item K is even. Then we can split the grid into two equal parts the following way:\\
		\begin{tabular}{CCCCCC}
			\gr{grid1} & \rightarrow & \gr{grid2} & \rightarrow & \gr{grid3} & \rightarrow\\
			\gr{grid4}& \rightarrow & \gr{grid5}	
		\end{tabular}		
		
		\begin{Def}
			Let the \textbf{center lines} be the shortest lines that are placed in-between two edges of the grid, e.g. the following grid has two of them (colored vertical lines). 
			
			\gr{grid4}
		\end{Def}		
		
		Here appears the invisible line splitting the grid in half. Bob's first move must be some vertex on a center line, because either way Riley can border the grid as shown on the next pictures:
		
		\begin{tabular}{CCCCCC}
			\gr{grid1} & \rightarrow & \gr{grid2_1} & \rightarrow & \gr{grid2_2.png} & \rightarrow\\
			\gr{grid2_3}
		\end{tabular}
		
		the horizontal lines in the previous example could be center lines, but the first move on them does not guarantee the optimal number of blue vertices.
		
		Therefore, in this case $B(G) = min(M, K) \cdot \dfrac{max(M, K)}{2}$, where $max(M, K)$ is even
		\item K is odd.
		
		\begin{itemize}
			\item M is odd. Then there's a center vertex. Bob's first move must be this vertex, I'll explain later why. We can draw four lines that separate the grid into parts that include the vertex and the parts that 
don't.

			\gr{b} \hfill \gr{b2}

			By acting symmetrically Riley can border one of these areas. We know that $M~and~K$ are odd, then $\exists m, k: M = 2m+1\quad K = 2k+1$. 
			
			The areas of the areas are $(2m+1)\cdot (k+1) ~~\&~~ (2k+1)\cdot (m+1)$, i.e. $2mk+k + 2m + 1~~\&~~ 2mk + m + 2k + 1$. 
			
			$M\leqslant K\Rightarrow m\leqslant k\Rightarrow m + 2mk + m + k + 1\leqslant  k + 2mk + m + k\Rightarrow $
			
			$\Rightarrow (2m+1)(k+1)\leqslant (2k+1)(m+1)$, so Riley will choose to take one of the $2mk + k$ areas. 
			
			if first Bob's move was not the central vertex, then several areas would expand, increasing the gain of Riley.
			
			Therefore $B(G) = min(M, K)\cdot \ceil{\dfrac{max(M, K)}{2}}$
			
			\item M is even. Then there's a single center line. Bob's first move must be on this center line.
			
			\gr{b3}
			
			And after his move Riley can make his moves symmetrically, taking the other half.
			
			If bob's first move was somewhere not on the center line, then Riley could border the grid so that he gets most of area.
			
			Therefore $B(G) = \ceil{\dfrac{min(M, K)}{2}}\cdot max(M, K),$ where $min(M, K)$ is even.
		\end{itemize}
	\end{itemize}
	
	
	
	Let's sum it up using the fact that $M,\leqslant K$ and make some inequalities:
	\begin{itemize}
		\item $K$ is even $\quad B(G) = M \cdot \ceil{\dfrac{K}{2}} = M\cdot \dfrac{K}{2}  = \dfrac{M}{2}\cdot K\leqslant \ceil{\dfrac{M}{2}}\cdot K$
		\item $K$ is odd:
		\begin{itemize}
			\item $M$ is even $\quad B(G) = \ceil{\dfrac{M}{2}}\cdot K = \dfrac{M}{2}\cdot K = M\cdot \dfrac{K}{2}\leqslant M\cdot \ceil{\dfrac{K}{2}}$	
			\item $M$ is odd $\quad B(G) = M\cdot \ceil{\dfrac{K}{2}}\leqslant \ceil{\dfrac{M}{2}}\cdot K$
			\begin{proof}
				$M, K$ are odd $\Rightarrow \begin{cases} M = 2m+1 & \ceil{\dfrac{M}{2}} = m+1\\K = 2k+1& \ceil{\dfrac{K}{2}} = k+1\\ \end{cases}\quad M\leqslant K\Rightarrow m\leqslant k$
				
				$m\leqslant k\Rightarrow 2mk + k + 2m + 1\leqslant 2mk + 2k + m + 1\Rightarrow (2m+1)(k+1)\leqslant (m+1)(2k+1) \Rightarrow M\ceil{\dfrac{K}{2}}\leqslant \ceil{\dfrac{M}{2}}\cdot K$
			\end{proof}
		\end{itemize}		 
	\end{itemize}
	
	We can conclude that $B(G) = min\left( \ceil{\dfrac{M}{2}}\cdot K~M\cdot \ceil{\dfrac{K}{2}}\right)$
	
\subsection{Torus Grid}

Since all vertices on a blank torus grid are completely identical, Bob can start his game with any vertex.

Let's consider the same cases. Suppose that $G$ is a torus grid $M\times K$, where $M\leqslant K$:
\begin{itemize}
	\item $K$ is even. Then there're two center lines. We can draw an invisible splitting line between them. Riley can make his moves symmetrically to Bob's moves. So Riley can guarantee that he will get at least half of the grid. So it's the maximum that Bob can get. $B(G) = M\cdot \dfrac{K}{2}$
	
	\gr{1_3}
	\item $K$ is odd
	\begin{itemize}
		\item $M$ is even. Then there's a line splitting the grid into two equal halves. And again Riley can act symmetrically
		
		\gr{1_31}
		\item $M$ is odd. There's a center vertex. Bob can start from it. As well as in the Grid case, we can draw four lines for symmetry.
		Riley can act symmetrically again, but there's a line that is symmetrical to itself. If they do their best to save their half and at the same time get that self-symmetrical line, they will split the graph with $\ceil{\dfrac{n}{2}}\to$ Bob and $\floor{\dfrac{n}{2}}\to$ Riley.
		
		\gr{b} 
	\end{itemize}
\end{itemize}  

\section{Upper and lower bounds for $M	_B(n)$}

Let's first consider $1\leqslant n\leqslant 5:$
\begin{tabular}{CCCCC}
	\gr{2}&\gr{2_2}&\gr{2_3}&\gr{2_4}&\gr{2_5}\\
\end{tabular}
\begin{enumerate}
	\item [$n=1$] there's only one 1-vertexed graph and $B(G) = 1$, so $M_B(n) = 1$
	\item [$n=2$] $M_B(n) = 1$
	\item [$n=3$] $M_B(G) = 2$. if all vertices are connected, then Bob can color any vertex for the first move. Riley chooses any of the left ones and Bob colors the third one. In cases where two vertices are not linked, there's a center vertex for Bob's first move and Riley can get only one of the remaining. And finally, bob colors another one. So $B(G) = 2$ for all cases.
	\item [$n=4$] $M_B(G) = 2$. You can't make $B(G)$ be 1, because there always exists a vertex with two connections. Riley can 'block' one but not all two, so $B(G)\geqslant 2\Rightarrow M_B(n) = 2$
	\item [$n=5$] $M_B(G) = 3$.  It can be easily shown that there's no 5-vertexed graph with $B(G)<3$
\end{enumerate}

\subsection{Upper bounds}

for the following $n$-s let's consider these three graphs (the numbers on vertices mean the number of invisible vertices that are connected only with this vertex):

$\begin{cases}
	n=3k&\Gr{2_6}\\
	n=3k-1&\Gr{2_61}\\\
	n=3k-2&\Gr{2_62}\\	
\end{cases}$

for $n = 3k$ in this graph $B(G) = 2 + k -2 = k = \dfrac{n}{3}$

for $n = 3k+1\quad B(G) = \ceil{\dfrac{n-6}{3}} +2 = \ceil{\dfrac{n}{3}} -2+2 = \ceil{\dfrac{n}{3}}$

for $n=3k+2\quad B(G) = \ceil{\dfrac{n}{3}}$

Therefore: $\begin{cases}
	M_B(n)\leqslant \ceil{\dfrac{n}{2}}, n<6\\
	M_B(n)\leqslant \ceil{\dfrac{n}{3}}, n\geqslant 6\\
		
\end{cases}$
\subsection{Lower bounds}
\section{Upper and lower bounds for $\lim\limits_{n\to\infty}\dfrac{M_B(n)}{n}$} To Be Done
\section*{New Rules}
For the next 3 problems, consider the following modification of the game rules. Instead of
choosing which vertices to color, each of the players has a pawn (blue for Bob and red for Riley)
that can move from vertex to vertex connected with an edge. A pawn can't step on a vertex with the opposite color. A pawn can step back on a vertex colored in its color.

Since the game (with the new rules) can continue indefinitely, for this case let $\p B (G)$ be the
maximal number of blue vertices Bob can guarantee in a finite amount of moves, and $\p R (G)$ be
the maximum number of red vertices Riley can guarantee in a finite amount of moves.

The other rules are the same:
\begin{enumerate}
	\item the first move can be any white vertex
	\item if there's no white vertices it's the end of the game
	\item you must step somewhere during your move unless there are no possible moves
\end{enumerate}	
\section{Value of $\p B(G) $ and $\p R(G)$}
\subsection{Grid}
Let $G$ be a grid $M\times K$.

\begin{itemize}
	\item $M$ is even or $K$ is even. The grid can be split into two halves by drawing a certain line. Then Riley can guarantee at least a half for him by moving symmetrically, so it's the maximum gain possible for Bob. $B(G) = \dfrac{M\cdot K}{2} = \ceil{\dfrac{M\cdot K}{2}}$
	\item $M$ and $K$ are both odd. Then there's a central vertex. If Bob's 1st move is not the central vertex, he gets less than a half. If he makes his first move right, he gets $\ceil{\dfrac{|G|}{2}} = \ceil{\dfrac{M\cdot K}{2}}$ (considering the case where both players make their moves to their benefit).
\end{itemize}

Therefore $B(G) = \ceil{\dfrac{M\cdot K}{2}}$

\subsection{Torus Grid}
Let $G$ be a torus grid $M\times K$

\begin{itemize}
	\item $M$ is even or $K$ is even. Then they can step symmetrically like in (4.1). And with the same reasons $B(G) = \dfrac{M\cdot K}{2} = \ceil{\dfrac{M\cdot K}{2}}$
	\item $M$ and $K$ are both odd. 
\end{itemize}
\end{document}











































